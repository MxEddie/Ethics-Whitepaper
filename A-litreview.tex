\section{Literature Review Methodology}
A primary goal of \textsc{Ethics Whitepaper} is to provide a comprehensive directory of resources for ethical research related to LLMs. As such, a systematic literature review of the ACL Anthology was conducted.\footnote{Note that \textsc{Ethics Whitepaper} is not itself strictly a systematic literature review as resources were added and removed subject to co-author evaluation.} The Anthology was searched for papers containing at least one term from the following key term lists in the abstract: related to the type of resource = \texttt{tool[a-z]*, toolkit, [A-za-z]*sheets*, guidelines*, principles, framework, approach}; related to ethics = \texttt{ethic[a-z]*, harms*, fair[a-z]*, risks*}. These lists were determined by first using more comprehensive lists then eliminating terms to improve the precision of the search. 
The resulting papers were manually reviewed to determine which were relevant to the scope of \textsc{Ethics Whitepaper}. During the search we identified a 2023 EACL tutorial titled ``Understanding Ethics in NLP Authoring and Reviewing'' \citep{benotti_understanding_2023}. The references for this tutorial were manually reviewed and where relevant included in \textsc{Ethics Whitepaper}. 

A second literature review was conducted using Semantic Scholar using the search terms:  \texttt{toolkit OR sheets OR guideline OR principles OR framework OR approach ethics OR ethical OR harms OR fair OR fairness OR risk AND "language models"}. 
These were likewise manually reviewed for inclusion.

The resulting resources were categorised by their relevance to different stages in a project's lifespan (from ideation to deployment). Primary themes in the literature were identified and used to structure the section. 

As \textsc{Ethics Whitepaper} progressed, additional resources familiar to the authors were added \textit{ad hoc}. Additionally, papers identified during the research review were removed if deemed no longer relevant - thus \textsc{Ethics Whitepaper} does not represent our systematic literature review in its entirety, but rather is primarily intended as a practical resource for conducting ethical research related to LLMs. Combining a literature review with our own expertise ensures broad coverage whilst maintaining a pragmatic focus. 